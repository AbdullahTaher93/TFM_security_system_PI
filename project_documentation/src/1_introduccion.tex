\fancypagestyle{miEstilo1}{
   \lhead{1. Introducción}
   %\chead{1. Introducción}
   \rhead{Página \thepage}
   \lfoot{}
   \cfoot{}
   \rfoot{}
}

\pagestyle{miEstilo1}

\section{Introducción}

\subsection{Motivación}

El número de robos y hurtos ha ido creciendo en estos últimos años, y por ello, es necesario la aplicación de medidas preventivas y correctoras que intenten reducir este número.

Hoy en día vivimos en una sociedad informatizada y tecnológica, donde cada uno tiene la capacidad de poder encontrar casi cualquier información allá donde esté, gracias a los dispositivos móviles y a la redes de comunicaciones que usa internet.

¿Quién no lleva un smartphone con conexión de datos en su bolsillo? En España el 85\% de la población hace uso de su smartphone a diario. Estos dispositivos móviles tienen una gran cantidad de usos y aplicaciones, y se han convertido en una herramienta imprescindible en nuestro día a día.

Uniendo tecnología y aplicación de medidas de seguridad basada en videovigilancia, surge la idea de este proyecto; la idea de construir un sistema de videovigilancia de bajo coste que sea fácil de usar, mantener, y lo más importante, que pueda ser controlado desde cualquier parte.

Partiendo de esta idea, se ha investigado y construido una aplicación que cumple con estas expectativas y más, ya que el uso que se le quiera dar puede ir más allá del concepto de videovigilancia para la seguridad. Por ejemplo, se puede utilizar para controlar entradas y salidas en una zona, control parental \ldots, aunque eso sí, dentro de un uso permitido y responsable.

Otro de los principales aspectos, objetivos y motivación de este proyecto, es que su uso no sea de forma privada, es decir, que pueda ser usado por todas las personas que lo deseen. Por este motivo, se ha intentado reducir todo el coste posible del hardware, y se ha liberado todo el código fuente, junto con las instrucciones necesarias para la instalación y despliegue de la aplicación.

Toda la información relacionada se puede consultar en el repositorio de github \cite{ref1}.

\subsection{Objetivos}

El objetivo principal de este proyecto es el de poder construir un sistema de seguridad basado en la videovigilancia, de bajo coste, y accesible a todo el mundo.

\textbf{Objetivos generales}

\vspace{-0.3cm}

\begin{itemize}
	\item Conseguir rebajar el coste de los sistemas de videovigilancia habituales.
	\item Alertar a un usuario ante un evento de movimiento generado dentro de una zona controlada.
	\item Almacenar y obtener pruebas (fotos y/o vídeo) tras la detección de cualquier intruso.
	\item Monitorizar y controlar el estado de un entorno.
	\item Lograr aumentar el nivel de seguridad de cualquier tipo de entorno.	
\end{itemize}

\textbf{Objetivos específicos}

\vspace{-0.3cm}

\begin{itemize}
	\item Desarrollar un sistema accesible para todo el mundo y fácil de usar.
	\item Permitir el acceso y gestión del sistema a través de un dispositivo móvil.
	\item Investigar y conocer el uso de los bots de telegram como medio de interacción entre el usuario y la aplicación back-end desarrollada en este proyecto.
	\item Estudiar el funcionamiento de una Raspberry PI y sus principales componentes para su aplicación en el ámbito del proyecto.
	\item Conocer, diseñar y hacer uso de una arquitectura basada en microservicios que permita el uso de servicios independientes, ágiles y escalables.
	\item Usar mecanismos de gestión de colas de mensajes y tareas asíncronas para evitar posibles esperas al interaccionar con la aplicación.
	\item Utilizar la inteligencia artificial para poder filtrar y evitar falsos positivos en las alertas generadas.

\end{itemize}

\newpage

\subsection{Estructura del documento}

Este documento está estructurado de la siguiente forma:

\textbf{1. Introducción}: Motivos y objetivos por los cuales se ha desarrollado este proyecto.

\textbf{2. Planificación del proyecto}: Muestra información detallada acerca de las diferentes etapas del proyecto, junto con la planificación prevista.

\textbf{3. Análisis de mercado}: Estudio comparativo acerca de software similar al propuesto en este proyecto.

\textbf{4. Tecnologías y herramientas utilizadas}: Breve descripción sobre las tecnologías y herramientas utilizadas para desarrollar este proyecto.

\newpage

