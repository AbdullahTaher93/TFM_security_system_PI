\thispagestyle{empty}

\begin{thebibliography}{99}

	\bibitem{ref1} Jonathan Martín Valera, TFM Security system PI github repository, disponible en \url{https://github.com/jmv74211/TFM_security_system_PI}

	\bibitem{ref2} Gus, 26 Jul 2019, Build a Raspberry Pi Security Camera Network,	disponible en \url{https://pimylifeup.com/raspberry-pi-security-camera/}
	
	\bibitem{ref3} ccrisan, 3 Sep 2019, A Video Surveillance OS For Single-board Computers ,	disponible en \url{https://github.com/ccrisan/motioneyeos}
	
	\bibitem{ref4} scavix, instructables.com, Raspberry Pi As Low-cost HD Surveillance Camera disponible en \url{https://www.instructables.com/id/Raspberry-Pi-as-low-cost-HD-surveillance-camera/}
	
	\bibitem{ref5} Motion-project, 19 Ago 2019, Camera video monitor, disponible en \url{https://github.com/Motion-Project/motion}
	
	\bibitem{ref6} Humberto Higinio, 14 Oct 2018, Sistema de Seguridad Raspberry Pi-Sensor de movimiento y cámara HD con envío de imágenes a correo, disponible en \url{https://www.youtube.com/watch?v=rK6uLwMRtIs}
	
	\bibitem{ref7} Google, DrawIO, diseño y diagramas de aplicaciones open source, disponible en \url{https://www.draw.io/}
	
	\bibitem{ref8} Visual paradigm, diseño de diagramas UML, disponible en \url{https://www.visual-paradigm.com/}
	
	\bibitem{ref9} Israel Alcázar, 03 Jun de 2014,  Introducción a Git y Github, disponible en \url{https://desarrolloweb.com/articulos/introduccion-git-github.html}
	
	\bibitem{ref10} Git, Sistema de control de versiones distribuido, multiplataforma y de código abierto, disponible en \url{https://git-scm.com/}
	
	\bibitem{ref11} Github, Servicio de alojamiento para el desarrollo de software utilizando Git , disponible en \url{https://git-scm.com/}
	
	\bibitem{ref12} PiCamera documentación, Interfaz de cámara Raspberry PI para python.
	
	\bibitem{ref13} pyTelegramBotAPI, Bibliotecas de funciones para la conexión con la API de telegram, disponible en \url{https://github.com/eternnoir/pyTelegramBotAPI}.
	
	\bibitem{ref14} Flask, framework para el despliegue ágil de aplicaciones sencillas, disponible en \url{https://flask.palletsprojects.com/en/1.1.x/}.
	
	\bibitem{ref15} Celery, Cola de tareas distribuidas, disponible en \url{http://www.celeryproject.org/}.
	
	\bibitem{ref16} RPi.GPIO, Biblioteca para controlar los pines GPIO de la Raspberry PI, disponible en \url{https://pypi.org/project/RPi.GPIO/}.
	
	\bibitem{ref17} Tensorflow, An end-to-end open source machine learning platform, disponible en \url{https://www.tensorflow.org/}.
	
	\bibitem{ref18} Requests, An end-to-end open source machine learning platform, disponible en \url{https://www.tensorflow.org/}.
	
	\bibitem{ref19} Latex-project, An introduction to LaTeX, disponible en \url{https://www.latex-project.org/about/}.

	\bibitem{ref20} TeX Live, 06 Jul de 2019, Introduction to TeX Live, disponible en \url{https://www.tug.org/texlive/}.
	
	\bibitem{ref21} TeXstudio.org, Welvome to TeXstudio, disponible en \url{https://www.texstudio.org/}.

	\bibitem{ref22} Jetbrains, Pycharm, The Python IDE
	for Professional Developers, disponible en \url{https://www.jetbrains.com/pycharm/}.
	
	\bibitem{ref23} RabbitMQ, Most widely deployed open source message broker, disponible en \url{https://www.rabbitmq.com/}.
	
	\bibitem{ref24} Telegram bot, Bots: An introduction for developers, disponible en \url{https://core.telegram.org/bots}.

	
\end{thebibliography}