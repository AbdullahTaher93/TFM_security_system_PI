

\fancypagestyle{miEstilo2}{
   \lhead{2. Planificación}
   %\chead{1. Introducción}
   \rhead{Página \thepage}
   \lfoot{}
   \cfoot{}
   \rfoot{}
}

\pagestyle{miEstilo2}

\section{Planificación}

Como en todo proyecto, realizar una buena planificación es la clave para no fracasar en el desarrollo del proyecto. Por ello, se ha realizado una planificación detallada del proyecto, utilizando el concepto de \textit{iteraciones} y \textit{sprints} para ir distribuyendo el trabajo a lo largo de las semanas, durante las cuales se han ido desarrollando una serie de tareas que cumplen un objetivo específico.

Estos conceptos surgen de los llamados métodos de desarrollo ágiles, en los cuales se intenta descomponer una aplicación o proyecto por funcionalidades, y el objetivo es ir desarrollando y testeando cada una de forma independientes para poder integrarla con el resto.

A continuación se muestra la planificación que se ha previsto para el desarrollo del proyecto.


\subsection{Fase preparatoria}

En esta fase se comenzará el desarrollo del proyecto. Tiene un esfuerzo estimado de X horas. Se realizará un estudio de viabilidad de la idea del proyecto, así como un estudio de mercado, una investigación acerca de las posibles herramientas a usar \ldots.

El objetivo de esta primera fase es tener claro que el proyecto es viable tanto en recursos, tiempo y presupuesto, para que posteriormente comience su desarrollo e implementación.

\large{\textbf{Iteración 1}: Análisis y estudio de la aplicación.}
\hrule

\vspace{0.3cm}

\normalsize

En esta iteración se describirán los principales objetivos de la aplicación, se estudiará sus posibles casos de uso y se elaborará un plan de iteraciones para planificar el desarrollo del proyecto a lo largo del tiempo dispuesto.

Esta iteración se divide en los siguientes sprints:

\textbf{Sprint 1}: Descripción de la aplicación



%\begin{center}
%\begin{tabular}{ | m{5em} | m{1cm}| m{1cm} | } 
%\hline
%cell1 dummy text dummy text dummy text& cell2 & cell3 \\ 
%\hline
%cell1 dummy text dummy text dummy text & cell5 & cell6 \\ 
%\hline
%cell7 & cell8 & cell9 \\ 
%\hline
%\end{tabular}
%\end{center}




\begin{table}[]
\begin{tabular}{|p{4cm}|p{7cm}|p{1.5cm}|p{2.5cm}|}
\hline
\rowcolor[HTML]{000000} 
{\color[HTML]{FFFFFF} Product backlog} & {\color[HTML]{FFFFFF} Descripción}                                  & {\color[HTML]{FFFFFF} Semana} & {\color[HTML]{FFFFFF} T.Previsión} \\ \hline
Objetivos                              & Descripción de los objetivos                                        & 1                             &                                        \\ \hline
Funcionalidades                        & Descripción de las funcionalidades                                  & 1                             &                                        \\ \hline
Motivaciones                           & Motivaciones personales acerca de la temática del proyecto          & 1                             &                                        \\ \hline
Usuarios y escenarios                  & Descripción de los posibles casos de uso de la aplicación propuesta & 1                             &                                        \\ \hline
\end{tabular}
\end{table}




